\documentclass{article}

\begin{document}

\section{Values, Atoms, Tags}

Values, ranged over by \(v\), are as in CompCert C.

\section{Memory Model}

Tagged C models memory as a flat address space in which each address \(a\)
has {\it contents} and {\it access permissions}. The contents of an address
is a pair of an atom and a tag

\[\mathit{con} \rightarrow A \rightharpoonup \]

\begin{tabular}{l l}
  
\end{tabular}

\subsection{The Allocation Oracle}

\subsection{Compiler Perturbations}

We call the oracle's perturb function when:
\begin{itemize}
\item 
\end{itemize}

\subsection{Permissions}

There are now only 3 permissions:
{\it Live}, representing allocated blocks which can always be accessed;
{\it MostlyDead}, representing memory that is not allocated but also is
not shadowed by the compiler; and
{\it Dead}, representing memory that is shadowed by the compiler.

Memory still carries a {\it nextblock} identifying the next block to be allocated,
and all blocks lower than it are considered valid blocks. We additionally keep track
of whether addresses are within a given block. A {\it valid access} is an offset that
is appropriately aligned and for which all elements have {\it Live} access; we require
that if 



\section{Axioms and Theorems}

As much as possible, I have tried to keep the existing memory theorems intact.
Some differences:

\begin{itemize}
\item We now need to use ``allowed access'' much more often than ``valid access,'' so
  
\end{itemize}

\end{document}
