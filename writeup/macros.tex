\newif\ifdraft \drafttrue
\newif\iftext \textfalse
\newif\iflater \latertrue
\newif\ifspace \spacefalse
\newif\ifaftersubmission \aftersubmissionfalse
\newif\ifcameraready  \camerareadyfalse
\newif\ifexceptions \exceptionsfalse

% !!! PLEASE DON'T CHANGE THESE !!! INSTEAD DEFINE YOUR OWN texdirectives.tex !!!
\makeatletter \@input{texdirectives} \makeatother

\newcommand{\gentag}[1][]{\mathit{\color{blue} t#1}}
\newcommand{\vt}[1][]{\mathit{\color{blue} vt#1}}
\newcommand{\pt}[1][]{\mathit{\color{blue} pt#1}}
\newcommand{\lt}[1][]{\mathit{\color{blue} lt#1}}
%\newcommand{\lts}{\overline{\color{blue} \lt}}
\newcommand{\nt}[1][]{\mathit{\color{blue} nt#1}}
\newcommand{\PCT}[1][]{\mathcal{\color{blue} P}\mathit{#1}}

\newcommand{\FN}[1][f]{{\color{blue} \textnormal{\sc f}_{#1}}}
\newcommand{\AN}[1][f,x]{{\color{blue} \textnormal{\sc a}_{#1}}}
\newcommand{\GN}[1][x]{{\color{blue} \textnormal{\sc g}_{#1}}}
\newcommand{\LN}[1][L]{{\color{blue} \textnormal{\sc l}_{#1}}}
\newcommand{\TN}[1][ty]{{\color{blue} \textnormal{\sc t}_{#1}}}

\newcommand{\apt}[1]{\ifdraft\textcolor{orange}{{[APT:~#1]}}\fi}
\newcommand{\sna}[1]{\ifdraft\textcolor{green!100}{{[SNA:~#1]}}\fi}
\newcommand{\amn}[1]{\ifdraft\textcolor{purple}{{AN:~#1}}\fi}

\newcommand{\deftag}{\mathit{\color{blue} def}}

%%% Policy Specific Tags %%%
\newcommand{\N}{{\color{blue} N}}
%% Initial Example %%
\newcommand{\PT}{\mathit{\color{blue} \AN[pwd]}}
\newcommand{\M}{{\color{blue} M}}
%% Memory Safety %%
\newcommand{\clr}{\color{blue} \mathit{clr}}
\newcommand{\tagz}{{\color{blue} 0}}
\newcommand{\tagone}{{\color{blue} 1}}
\newcommand{\tagtwo}{{\color{blue} 2}}
%% SIF %%
\newcommand{\high}{\textnormal{\color{blue} \sc h}}
\newcommand{\low}{\textnormal{\color{blue} \sc l}}
\newcommand{\vtaint}[1]{\mathit{\color{blue} vtaint ~ #1}}
\newcommand{\pctaint}[3]{\textnormal{\sc \color{blue} pc} \mathit{\color{blue} ~ #1 ~ #2 ~ #3}}
\newcommand{\vol}{\mathit{\color{blue} vol}}
\newcommand{\sink}[1]{\mathit{\color{blue} sink ~ #1}}

\newcommand{\fail}{\mathbf{\color{red} fail}}

\newcommand{\trule}[2]{{\color{blue} #1 \leftarrow #2}}

\newcommand{\truledef}[1]{
  & \multispan{3} \({\color{blue} #1}\) \\}

\newcommand{\assert}[1]{& & & \multispan{2} \(\mathbf{assert} ~ #1\) \hfill \\}
\newcommand{\letin}[1]{& & & \multispan{2} \(\mathit{let} ~ #1 ~ \mathit{in}\) \\}

\newcommand{\settag}[2]{\boldsymbol{#1} & \longleftarrow & & \mathit{#2}\\}
\newcommand{\settagopt}[2]{\optional{\(\boldsymbol{#1}\)} & \longleftarrow & & \mathit{#2}\\}


%%% Continuations, States, Values %%%

\newcommand{\kemp}{\mathit{Kemp}}
\newcommand{\kdo}[1]{\mathit{Kdo};~ #1}
\newcommand{\kseq}[2]{\mathit{Kseq} ~ #1; ~ #2}
\newcommand{\kif}[4]{\mathit{Kif} ~ #1 ~ #2 ~ #3; ~ #4}
\newcommand{\kwhiletest}[4]{\mathit{KwhileTest} ~ #1 ~ #2 ~ #3; ~ #4}
\newcommand{\kwhileloop}[4]{\mathit{KwhileLoop} ~ #1 ~ #2 ~ #3; ~ #4}
\newcommand{\kdowhiletest}[4]{\mathit{KdoWhileTest} ~ #1 ~ #2 ~ #3; ~ #4}
\newcommand{\kdowhileloop}[4]{\mathit{KdoWhileLoop} ~ #1 ~ #2 ~ #3; ~ #4}
\newcommand{\kfor}[5]{\mathit{Kfor} ~ (#1,#2) ~ #3 ~ #4 ; ~ #5}
\newcommand{\kforpost}[5]{\mathit{KforPost} ~ (#1,#2) ~ #3 ~ #4 ; ~ #5}
\newcommand{\kcall}[4]{\mathit{Kcall} ~ #1 ~ #2 ~ #3; ~ #4}
\newcommand{\kreturn}[1]{\mathit{Kreturn}; ~ #1}
\newcommand{\kswitcha}[3]{\mathit{Kswitch1} ~ #1 ~ #2; ~ #3}
\newcommand{\kswitchb}[1]{\mathit{Kswitch2}; ~ #1}

\newcommand{\ctx}[1]{C \!\left[#1\right]}
\newcommand{\rh}{\textnormal{\sc rh}}
\newcommand{\lh}{\textnormal{\sc lh}}

\newcommand{\sstate}[4]{\mathcal{S}\left(#2 \mid #3 \gg #4 {\color{blue} @ #1}\right)}
\newcommand{\estate}[4]{\mathcal{E}\left(#2 \mid #3 \gg #4 {\color{blue} @ #1}\right)}
\newcommand{\cstate}[5]{\mathcal{C}\left(#2 \mid #3(#4) \gg #5 {\color{blue} @ #1}\right)}
\newcommand{\rstate}[4]{\mathcal{R}\left(#2 \mid #3 \gg #4 {\color{blue} @ #1}\right)}
\newcommand{\fstate}[1]{\mathcal{F}\left(#1\right)}
\newcommand{\sstatefun}[5]{\mathcal{S}\left(#1, #3 \mid #4 \gg #5 {\color{blue} @ #2}\right)}

\newcommand{\mem}{m}
\newcommand{\genv}{\mathit{ge}}
\newcommand{\lenv}{\mathit{le}}
\newcommand{\cont}{k}
\newcommand{\stmt}{s}
\newcommand{\expr}{e}
\newcommand{\type}{ty}
\newcommand{\defestate}[1]
           {\estate{\PCT}{\mem}{#1}{\cont}}
\newcommand{\defsstate}[1]
           {\sstate{\PCT}{\mem}{#1}{\cont}}

%% Expressions $$
\newcommand{\val}[2]{\mathit{Eval} ~ #1@#2}
\newcommand{\var}[1]{\mathit{Evar} ~ #1}
\newcommand{\field}[2]{\mathit{Efield} ~ #1 ~ #2}
\newcommand{\valof}[1]{\mathit{EvalOf} ~ #1}
\newcommand{\deref}[1]{\mathit{Ederef} ~ #1}
\newcommand{\addrof}[1]{\mathit{EaddrOf} ~ #1}
\newcommand{\unop}[2]{\mathit{Eunop} ~ #1 ~ #2}
\newcommand{\binop}[3]{\mathit{Ebinop} ~ #1 ~ #2 ~ #3}
\newcommand{\cast}[2]{\mathit{Ecast} ~ #1 ~ #2}
\newcommand{\seqand}[2]{\mathit{EseqAnd ~ #1 ~ #2}}
\newcommand{\seqor}[2]{\mathit{EseqOr ~ #1 ~ #2}}
\newcommand{\condition}[3]{\mathit{Econd} ~ #1 ~ #2 ~ #3}
\newcommand{\sizeof}[1]{\mathit{Esize} ~ #1}
\newcommand{\alignof}[1]{\mathit{Ealign} ~ #1}
\newcommand{\assign}[2]{\mathit{Eassign ~ #1 ~ #2}}
\newcommand{\assignop}[3]{\mathit{EassignOp ~ #1 ~ #2 ~ #3}}
\newcommand{\postinc}[2]{\mathit{EpostInc} ~ #1 ~ #2}
\newcommand{\comma}[2]{\mathit{Ecomma} ~ #1 ~ #2}
\newcommand{\call}[2]{\mathit{Ecall} ~ #1(#2)}
\newcommand{\builtin}[2]{\mathit{Ebuiltin} ~ #1(#2)}
\newcommand{\loc}[2]{\mathit{Eloc} ~ #1@#2}
\newcommand{\floc}[2]{\mathit{Efloc} ~ #1@#2}
\newcommand{\paren}[2]{\mathit{Eparen} ~ #1 ~ #2}
\newcommand{\failstop}{\mathit{Efailstop}}

\newcommand{\sskip}{\mathtt{Sskip}}
\newcommand{\sdo}[1]{\mathtt{Sdo} ~ #1}
\newcommand{\sseq}[2]{\mathtt{Sseq} ~ #1 ~ #2}
\newcommand{\sifthenelse}[4]{\mathtt{Sif}(#1) ~ \mathtt{then} ~ #2 ~ \mathtt{else} ~ #3 ~ \mathtt{join} ~ #4}
\newcommand{\swhile}[3]{\mathtt{Swhile}(#1) ~ \mathtt{do} ~ #2 ~ \mathtt{join} ~ #3}
\newcommand{\sdowhile}[3]{\mathtt{Sdo} ~ #2 ~ \mathtt{while} ~ (#1) ~ \mathtt{join} ~ #3}
\newcommand{\sfor}[5]{\mathtt{Sfor}(#1; #2; #3) ~ \mathtt{do} ~ #4 ~ \mathtt{join} ~ #5}
\newcommand{\sbreak}{\mathtt{Sbreak}}
\newcommand{\scontinue}{\mathtt{Scontinue}}
\newcommand{\sreturn}[1]{\mathtt{Sreturn} ~ #1}
\newcommand{\sswitch}[3]{\mathtt{Sswitch} ~ #1 ~ \{ ~ #2 ~ \} ~ \mathtt{join} ~ #3}
\newcommand{\slabel}[2]{\mathtt{Slabel} ~ #1: ~ #2}
\newcommand{\sgoto}[1]{\mathtt{Sgoto} ~ #1}

\newcommand{\vundef}{\mathbf{undef}}

\newcommand{\tptr}[1]{\mathit{ptr(#1)}}

\newcommand{\judgment}[3][]{
  {\centering
  \smallskip
  \begin{tabular}{c}
    #2 \\
    \hline
    #3
  \end{tabular}{\sc #1}
  \smallskip\par}}

\newcommand{\judgmentbr}[4][]{
  {\centering
  \smallskip
  \begin{tabular}{c}
    #2 \\
    #3 \\
    \hline
    #4
  \end{tabular}{\sc #1}
   \smallskip\par}}

\newcommand{\judgmentbrbr}[5][]{
  {\centering
  \smallskip
  \begin{tabular}{c}
    #2 \\
    #3 \\
    #4 \\
    \hline
    #5
  \end{tabular}{\sc #1}
   \smallskip\par}}

\newcommand{\judgmentbrbrbr}[6][]{
  {\centering
  \smallskip
  \begin{tabular}{c}
    #2 \\
    #3 \\
    #4 \\
    #5 \\
    \hline
    #6
  \end{tabular}{\sc #1}
   \smallskip\par}}

\newcommand{\judgmenttwobr}[6][]{
  {
    \centering
    \smallskip
    \begin{tabular}{c c}
       #2 & #3 \\
       #4 & #5 \\
       \hline
       \multicolumn{2}{c}{#6}
    \end{tabular}{\sc #1}
    \vspace{\belowdisplayskip}\par
  }}

\newcommand{\judgmenttwobrlong}[5][]{
  {
    \centering
    \smallskip
    \begin{tabular}{c c}
       #2 & #3 \\
       \multicolumn{2}{c}{#4} \\
       \hline
       \multicolumn{2}{c}{#5}
    \end{tabular}{\sc #1}
    \vspace{\belowdisplayskip}\par
  }}

\newcommand{\judgmentthreebrlong}[6][]{
  {
    \centering
    \smallskip
    \begin{tabular}{c c c}
       #2 & #3 & #4 \\
       \multicolumn{3}{c}{#5} \\
       \hline
       \multicolumn{3}{c}{#6}
    \end{tabular}{\sc #1}
    \vspace{\belowdisplayskip}\par
  }}

\newcommand{\judgmentthreebrtwo}[7][]{
  {
    \centering
    \smallskip
    \begin{tabular}{c c c}
       #2 & #3 & #4 \\
       \multicolumn{3}{c}{#5 \hfill #6} \\
       \hline
       \multicolumn{3}{c}{#7}
    \end{tabular}{\sc #1}
    \vspace{\belowdisplayskip}\par
  }}

\newcommand{\judgmenttwobrlongbrlong}[6][]{
  {
    \centering
    \smallskip
    \begin{tabular}{c c}
       #2 & #3 \\
       \multicolumn{2}{c}{#4} \\
       \multicolumn{2}{c}{#5} \\
       \hline
       \multicolumn{2}{c}{#6} \\
    \end{tabular}{\sc #1}
    \vspace{\belowdisplayskip}\par
  }}


\newcommand{\judgmentthreebr}[8][]{
  {
    \centering
    \smallskip
    \begin{tabular}{c c c}
       #2 & #3 & #4 \\
       #5 & #6 & #7 \\
       \hline
       \multicolumn{3}{c}{#8}
    \end{tabular}{\sc #1}
    \vspace{\belowdisplayskip}\par
  }}


\newcommand{\judgmenttwo}[4][]{
  {\centering
  \smallskip
  \begin{tabular}{c c}
    #2 & #3 \\
    \hline
    \multicolumn{2}{c}{#4}
  \end{tabular}{\sc #1}
  \smallskip\par}}

\newcommand{\judgmentthree}[5][]{
  {\centering
  \smallskip
  \begin{tabular}{c c c}
    #2 & #3 & #4 \\
    \hline
    \multicolumn{3}{c}{#5}
  \end{tabular}{\sc #1}
  \smallskip\par}}

\newcommand{\judgmentfour}[6][]{
  {\centering
  \smallskip
  \begin{tabular}{c c c c}
    #2 & #3 & #4 & #5 \\
    \hline
    \multicolumn{4}{c}{#6}
  \end{tabular}{\sc #1}
  \smallskip\par}}
