\newcommand{\tagcolor}{C}

\newcommand{\vt}{\mathit{vt}}
\newcommand{\pt}{\mathit{pt}}
\newcommand{\lt}{\mathit{lt}}
\newcommand{\lts}{\overline{\lt}}
\newcommand{\nt}{\mathit{nt}}
\newcommand{\PCT}{\mathcal{P}}

\newcommand{\trule}[2]{#1 \leftarrow #2}

\newcommand{\truledef}[1]{
  & \multispan{3} \(#1\) \\}

\newcommand{\assert}[1]{& & & \multispan{2} \(\mathbf{assert} ~ #1\) \hfill \\}
\newcommand{\letin}[1]{& & & \multispan{2} \(\mathit{let} ~ #1 ~ \mathit{in}\) \\}

\newcommand{\caseof}[1]{\textnormal{case } #1 \textnormal{ of}}
\newcommand{\caseentry}[2]{& & & #1 \Rightarrow #2}

\newcommand{\optional}[1]{\fcolorbox{black}{gray!20}{#1}}
\newcommand{\settag}[2]{\boldsymbol{#1} & \longleftarrow & & \mathit{#2}\\}
\newcommand{\settagopt}[2]{\optional{\(\boldsymbol{#1}\)} & \longleftarrow & & \mathit{#2}\\}

%%% Tag Rules %%%
\newcommand{\loadtname}{\mathbf{LoadT}}
\newcommand{\loadtargs}{\PCT, \pt, \vt, \overline{\lt}}
\newcommand{\loadtres}{\vt'}
\newcommand{\loadt}{\loadtname(\loadtargs)}

\newcommand{\storetname}{\mathbf{StoreT}}
\newcommand{\storetargs}{\PCT, \pt, \vt_1, \vt_2, \overline{\lt}}
\newcommand{\storetres}{\PCT',\vt',\overline{\lt}'}
\newcommand{\storet}{\storetname(\storetargs)}

\newcommand{\consttname}{\mathbf{ConstT}}
\newcommand{\consttres}{\vt}
\newcommand{\constt}{\consttname}

\newcommand{\unoptname}{\mathbf{UnopT}}
\newcommand{\unoptargs}{\PCT, \vt}
\newcommand{\unoptres}{\vt}
\newcommand{\unopt}{\unoptname(\unoptargs)}

\newcommand{\binoptname}{\mathbf{BinopT}}
\newcommand{\binoptargs}{\PCT, \vt_1, \vt_2}
\newcommand{\binoptres}{\vt'}
\newcommand{\binopt}{\binoptname(\binoptargs)}

\newcommand{\globaltname}{\mathbf{GlobalT}}
\newcommand{\globaltargs}{id, s}
\newcommand{\globaltargstyped}{id \in ident, s \in \mathbb{N}}
\newcommand{\globaltres}{\pt,\vt,\overline{\lt}}
\newcommand{\globalt}{\globaltname(\globaltargs)}

\newcommand{\localtname}{\mathbf{LocalT}}
\newcommand{\localtargs}{\PCT, id, s}
\newcommand{\localtargstyped}{\PCT, id \in ident, s \in \mathbb{N}}
\newcommand{\localtres}{\pt,\vt,\overline{\lt}}
\newcommand{\localt}{\localtname(\localtargs)}

\newcommand{\vartname}{\mathbf{VarT}}
\newcommand{\vartargs}{\PCT, \vt}
\newcommand{\vartres}{\pt}
\newcommand{\vart}{\vartname(\vartargs)}

\newcommand{\malloctname}{\mathbf{MallocT}}
\newcommand{\malloctargs}{\PCT, \vt}
\newcommand{\malloctres}{\PCT',\pt,\optional{\(\vt,\overline{\lt}\)}}
\newcommand{\malloct}{\malloctname(\malloctargs)}

\newcommand{\freetname}{\mathbf{FreeT}}
\newcommand{\freetargs}{\PCT, \vt}
\newcommand{\freetres}{\PCT',\pt,\optional{\(\vt,\overline{\lt}\)}}
\newcommand{\freet}{\freetname(\freetargs)}

\newcommand{\picasttname}{\mathbf{PICastT}}
\newcommand{\picasttargs}{\PCT, \pt, \optional{\(\vt, \overline{\lt}\)}}
\newcommand{\picasttres}{\PCT',\vt}
\newcommand{\picastt}{\picasttname(\picasttargs)}

\newcommand{\ipcasttname}{\mathbf{IPCastT}}
\newcommand{\ipcasttargs}{\PCT, \vt_1, \optional{\(\vt_2, \overline{\lt}\)}}
\newcommand{\ipcasttres}{\PCT',\pt}
\newcommand{\ipcastt}{\ipcasttname(\ipcasttargs)}

\newcommand{\ppcasttname}{\mathbf{PPCastT}}
\newcommand{\ppcasttargs}{\PCT, \pt, \optional{\(\vt, \overline{\lt}\)}}
\newcommand{\ppcasttres}{\PCT',\pt'}
\newcommand{\ppcastt}{\picasttname(\picasttargs)}

\newcommand{\iicasttname}{\mathbf{IICastT}}
\newcommand{\iicasttargs}{\PCT, \vt_1}
\newcommand{\iicasttres}{\PCT',\pt}
\newcommand{\iicastt}{\ipcasttname(\ipcasttargs)}

\newcommand{\splittname}{\mathbf{SplitT}}
\newcommand{\splittargs}{\PCT, \vt, \optional{\(L\)}}
\newcommand{\splittres}{\PCT'}
\newcommand{\splitt}{\splittname(\splittargs)}
           
\newcommand{\jointname}{\mathbf{JointT}}
\newcommand{\jointargs}{\PCT, \optional{\(L\)}}
\newcommand{\jointres}{\PCT'}
\newcommand{\joint}{\jointname(\jointargs)}

\newcommand{\argtname}{\mathbf{ArgT}}
\newcommand{\argtargs}{\PCT, \vt, f, x}
\newcommand{\argtargstyped}{\PCT, \vt, f, x \in ident}
\newcommand{\argtres}{\vt'}
\newcommand{\argt}{\argtname(\argtargs)}

\newcommand{\callerrettname}{\mathbf{CallerRetT}}
\newcommand{\callerrettargs}{\PCT, \PCT', \vt}
\newcommand{\callerrettres}{\vt'}
\newcommand{\callerrett}{\callerrettname(\callerrettargs)}

\newcommand{\calleerettname}{\mathbf{CalleeRetT}}
\newcommand{\calleerettargs}{\PCT, \PCT', \vt}
\newcommand{\calleerettres}{\vt'}
\newcommand{\calleerett}{\calleerettname(\calleerettargs)}

%%%%%%%%%%%%%%%%%

%%% Continuations, States, Values %%%

\newcommand{\kemp}{\mathit{Kemp}}
\newcommand{\kdo}[1]{\mathit{Kdo};~ #1}
\newcommand{\kseq}[2]{\mathit{Kseq} ~ #1; ~ #2}
\newcommand{\kif}[4]{\mathit{Kif}[#1 \mid #2] ~ \mathit{join} ~ #3; ~ #4}
\newcommand{\kwhiletest}[4]{\mathit{KwhileTest}(#1) ~ \{ ~ #2 ~ \} ~ \mathit{join} ~ #3; ~ #4}
\newcommand{\kwhileloop}[4]{\mathit{KwhileLoop}(#1) ~ \{ ~ #2 ~ \} ~ \mathit{join} ~ #3; ~ #4}
\newcommand{\kdowhiletest}[4]{\mathit{KdoWhileTest}(#1) ~ \{ ~ #2 ~ \} ~ \mathit{join} ~ #3; ~ #4}
\newcommand{\kdowhileloop}[4]{\mathit{KdoWhileLoop}(#1) ~ \{ ~ #2 ~ \} ~ \mathit{join} ~ #3; ~ #4}
\newcommand{\kfor}[2]{\mathit{Kfor} ~ #1; ~ #2}
\newcommand{\kforpost}[2]{\mathit{KforPost} ~ #1; ~ #2}
\newcommand{\kcall}[3]{\mathit{Kcall} ~ #1 ~ #2 ~ #3}

\newcommand{\ctx}[1]{ctx \left[#1\right]}

\newcommand{\sstate}[6]{\mathcal{S}\left(f,#2,#3,#4 \mid #5 \gg #6 @ #1\right)}
\newcommand{\estate}[6]{\mathcal{E}\left(f,#2,#3,#4 \mid #5; \gg #6 @ #1\right)}
\newcommand{\cstate}[7]{\mathcal{C}\left(#1,#3,#4 \mid #5(#6) \gg #7 @ #2\right)}
\newcommand{\rstate}[6]{\mathcal{R}\left(#1,#3,#4 \mid #5 \gg #6 @ #2\right)}
\newcommand{\fstate}[1]{\mathcal{F}\left(#1\right)}

\newcommand{\mem}{m}
\newcommand{\genv}{\mathit{ge}}
\newcommand{\lenv}{\mathit{le}}
\newcommand{\cont}{k}
\newcommand{\stmt}{s}
\newcommand{\expr}{e}
\newcommand{\type}{ty}
\newcommand{\defestate}[1]
           {\estate{\PCT}{\mem}{\genv}{\lenv}{#1}{\cont}}
\newcommand{\defsstate}[1]
           {\sstate{\PCT}{\mem}{\genv}{\lenv}{#1}{\cont}}

\newcommand{\valof}[1]{|#1|}
\newcommand{\deref}[1]{* #1}
\newcommand{\addrof}[1]{\& #1}
\newcommand{\assignop}[3]{#2 ~~ [#1]\!\!= #3}
\newcommand{\postinc}[2]{#2 #1\!\!#1}
\newcommand{\assign}[2]{#1 := #2}
\newcommand{\loc}[2]{\underline{#1}@#2}
\newcommand{\val}[2]{\mathit{#1} @ #2}
\newcommand{\binop}[3]{#2 #1 #3}
\newcommand{\unop}[2]{#1 #2}
\newcommand{\comma}[2]{#1, #2}
\newcommand{\paren}[2]{(#2) (#1)}
\newcommand{\builtin}[2]{\mathit{builtin} ~ #1(#2)}
\newcommand{\var}[1]{#1}
\newcommand{\cast}[2]{(#2) #1}
\newcommand{\call}[2]{#1(#2)}
\newcommand{\condition}[3]{#1 ~ ? ~ #2 ~ : ~ #3}
\newcommand{\sizeof}[1]{\mathtt{size}(#1)}
\newcommand{\alignof}[1]{\mathtt{align}(#1)}

\newcommand{\sskip}{\mathtt{skip}}
\newcommand{\sdo}[1]{#1;}
\newcommand{\sseq}[2]{#1 ~ #2}
\newcommand{\scontinue}{\mathtt{continue}}
\newcommand{\sbreak}{\mathtt{break}}
\newcommand{\sreturn}{\mathtt{return}}
\newcommand{\sifthenelse}[4]{\mathtt{if}(#1) ~ \mathtt{then} ~ #2 ~ \mathtt{else} ~ #3 ~ \mathtt{join} ~ #4}
\newcommand{\swhile}[3]{\mathtt{while}(#1) ~ \mathtt{do} ~ #2 ~ \mathtt{join} ~ #3}
\newcommand{\sdowhile}[3]{\mathtt{do} ~ #2 ~ \mathtt{while} ~ (#1) ~ \mathtt{join} ~ #3}
\newcommand{\sfor}[5]{\mathtt{for}(#1; #2; #3) ~ \mathtt{do} ~ #4 ~ \mathtt{join} ~ #5}
\newcommand{\sswitch}[2]{\mathtt{switch} ~ #1 ~ \{ ~ #2 ~ \}}
\newcommand{\slabel}[2]{#1: ~ #2}
\newcommand{\sgoto}[1]{\mathtt{goto} ~ #1}

\newcommand{\vundef}{\mathbf{undef}}

\newcommand{\tptr}[1]{\mathit{ptr(#1)}}

\newcommand{\judgment}[3][]{
  {\centering
  \smallskip
  \begin{tabular}{c}
    #2 \\
    \hline
    #3
  \end{tabular}{\sc #1}
  \smallskip\par}}

\newcommand{\judgmentbr}[4][]{
  {\centering
  \smallskip
  \begin{tabular}{c}
    #2 \\
    #3 \\
    \hline
    #4
  \end{tabular}{\sc #1}
   \smallskip\par}}

\newcommand{\judgmentbrbr}[5][]{
  {\centering
  \smallskip
  \begin{tabular}{c}
    #2 \\
    #3 \\
    #4 \\
    \hline
    #5
  \end{tabular}{\sc #1}
   \smallskip\par}}

\newcommand{\judgmentbrbrbr}[6][]{
  {\centering
  \smallskip
  \begin{tabular}{c}
    #2 \\
    #3 \\
    #4 \\
    #5 \\
    \hline
    #6
  \end{tabular}{\sc #1}
   \smallskip\par}}

\newcommand{\judgmenttwobr}[6][]{
  {
    \centering
    \smallskip
    \begin{tabular}{c c}
       #2 & #3 \\
       #4 & #5 \\
       \hline
       \multicolumn{2}{c}{#6}
    \end{tabular}{\sc #1}
    \vspace{\belowdisplayskip}\par
  }}

\newcommand{\judgmenttwobrlong}[5][]{
  {
    \centering
    \smallskip
    \begin{tabular}{c c}
       #2 & #3 \\
       \multicolumn{2}{c}{#4} \\
       \hline
       \multicolumn{2}{c}{#5}
    \end{tabular}{\sc #1}
    \vspace{\belowdisplayskip}\par
  }}

\newcommand{\judgmentthreebrlong}[6][]{
  {
    \centering
    \smallskip
    \begin{tabular}{c c c}
       #2 & #3 & #4 \\
       \multicolumn{3}{c}{#5} \\
       \hline
       \multicolumn{3}{c}{#6}
    \end{tabular}{\sc #1}
    \vspace{\belowdisplayskip}\par
  }}

\newcommand{\judgmentthreebrtwo}[7][]{
  {
    \centering
    \smallskip
    \begin{tabular}{c c c}
       #2 & #3 & #4 \\
       \multicolumn{3}{c}{#5 \hfill #6} \\
       \hline
       \multicolumn{3}{c}{#7}
    \end{tabular}{\sc #1}
    \vspace{\belowdisplayskip}\par
  }}

\newcommand{\judgmenttwobrlongbrlong}[6][]{
  {
    \centering
    \smallskip
    \begin{tabular}{c c}
       #2 & #3 \\
       \multicolumn{2}{c}{#4} \\
       \multicolumn{2}{c}{#5} \\
       \hline
       \multicolumn{2}{c}{#6} \\
    \end{tabular}{\sc #1}
    \vspace{\belowdisplayskip}\par
  }}


\newcommand{\judgmentthreebr}[8][]{
  {
    \centering
    \smallskip
    \begin{tabular}{c c c}
       #2 & #3 & #4 \\
       #5 & #6 & #7 \\
       \hline
       \multicolumn{3}{c}{#8}
    \end{tabular}{\sc #1}
    \vspace{\belowdisplayskip}\par
  }}


\newcommand{\judgmenttwo}[4][]{
  {\centering
  \smallskip
  \begin{tabular}{c c}
    #2 & #3 \\
    \hline
    \multicolumn{2}{c}{#4}
  \end{tabular}{\sc #1}
  \smallskip\par}}

\newcommand{\judgmentthree}[5][]{
  {\centering
  \smallskip
  \begin{tabular}{c c c}
    #2 & #3 & #4 \\
    \hline
    \multicolumn{3}{c}{#5}
  \end{tabular}{\sc #1}
  \smallskip\par}}

\newcommand{\judgmentfour}[6][]{
  {\centering
  \smallskip
  \begin{tabular}{c c c c}
    #2 & #3 & #4 & #5 \\
    \hline
    \multicolumn{4}{c}{#6}
  \end{tabular}{\sc #1}
  \smallskip\par}}
