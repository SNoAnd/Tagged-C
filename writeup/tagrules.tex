\newcommand{\truleblock}[2]
           {
             \colorbox{blue!10}{
             \begin{tabular}{l}
               #1 \\
               #2 \\
             \end{tabular}}
           }

\newcommand{\malloctname}{\color{blue} \mathbf{MallocT}}
\newcommand{\malloctargs}{\PCT, \FN, \FN[f'], \vt}
\newcommand{\malloctres}{\PCT',\pt,\vt,\overline{\lt}}
\newcommand{\malloct}{\malloctname(\malloctargs)}

\newcommand{\malloctruleblock}[1]
           {
             \truleblock{\(\malloct\)}{#1}
           }

\newcommand{\localtname}{\color{blue} \mathbf{LocalT}}
\newcommand{\localtargs}{\PCT, \TN}
\newcommand{\localtres}{\PCT', \pt,\vt,\overline{\lt}}
\newcommand{\localt}{\localtname(\localtargs)}

\newcommand{\localtruleblock}[1]
           {
             \truleblock{\(\localt\)}{#1} \\
           }

\newcommand{\accesstname}{\color{blue} \mathbf{AccessT}}
\newcommand{\accesstargs}{\PCT,\vt}
\newcommand{\accesstres}{\vt'}
\newcommand{\accesst}{\accesstname(\accesstargs)}

\newcommand{\accesstruleblock}[1]
           {
             \truleblock{\(\accesst\)}{#1}
           }

\newcommand{\loadtname}{\color{blue} \mathbf{LoadT}}
\newcommand{\loadtargs}{\PCT, \pt, \vt, \overline{\lt}}
\newcommand{\loadtres}{\vt'}
\newcommand{\loadt}{\loadtname(\loadtargs)}

\newcommand{\loadtruleblock}[1]
           {
             \truleblock{\(\loadt\)}{#1}
           }

\newcommand{\assigntname}{\color{blue} \mathbf{AssignT}}
\newcommand{\assigntargs}{\PCT,\vt_1,\vt_2}
\newcommand{\assigntres}{\PCT',\vt'}
\newcommand{\assignt}{\assigntname(\assigntargs)}

\newcommand{\assigntruleblock}[1]
           {
             \truleblock{\(\assignt\)}{#1}
           }
           
\newcommand{\storetname}{\color{blue} \mathbf{StoreT}}
\newcommand{\storetargs}{\PCT, \pt, \vt, \lt}
\newcommand{\storetres}{\PCT',\vt',\lt'}
\newcommand{\storet}{\storetname(\storetargs)}

\newcommand{\storetruleblock}[1]
           {
             \truleblock{\(\storet\)}{#1}
           }

\newcommand{\unoptname}{\color{blue} \mathbf{UnopT}}
\newcommand{\unoptargs}{\odot, \PCT, \vt}
\newcommand{\unoptres}{\vt}
\newcommand{\unopt}{\unoptname(\unoptargs)}
           
\newcommand{\unoptruleblock}[2]
           {
             \colorbox{blue!10}{
               \begin{tabular}{l}
                 \(\unopt\) \\
                 % body
                 #1 \\
                 % outputs
                 \{ \(\unoptres\) \} \\
               \end{tabular}
             }
           }
           
\newcommand{\binoptname}{\color{blue} \mathbf{BinopT}}
\newcommand{\binoptargs}{\oplus, \PCT, \vt_1, \vt_2}
\newcommand{\binoptres}{\vt'}
\newcommand{\binopt}{\binoptname(\binoptargs)}

\newcommand{\binoptruleblock}[1]
           {
             \truleblock{\(\binopt\)}{#1}
           }

\newcommand{\calltname}{\color{blue} \mathbf{CallT}}
\newcommand{\calltargs}{\PCT, \FN, \FN[f']}
\newcommand{\calltres}{\PCT'}
\newcommand{\callt}{\calltname(\calltargs)}
           
\newcommand{\calltruleblock}[1]
           {
             \colorbox{blue!10}{
               \begin{tabular}{l}
                 \(\callt\) \\
                 % body
                 #1 \\
                 % outputs
                 \{ \(\calltres\) \} \\
               \end{tabular}
             }
           }

\newcommand{\argtname}{\color{blue} \mathbf{ArgT}}
\newcommand{\argtargs}{\PCT, \vt, \AN, \TN}
\newcommand{\argtres}{\PCT', \pt, \vt',\lts}
\newcommand{\argt}{\argtname(\argtargs)}

\newcommand{\argtruleblock}[1]
           {
             \truleblock{\(\argt\)}{#1}
           }
           
\newcommand{\rettname}{\color{blue} \mathbf{RetT}}
\newcommand{\rettargs}{\PCT_{CLE}, \PCT_{CLR}, \vt, \FN}
\newcommand{\rettres}{\PCT',\vt'}
\newcommand{\rett}{\rettname(\rettargs)}

\newcommand{\rettruleblock}[1]
           {
             \colorbox{blue!10}{
               \begin{tabular}{l}
                 \(\rett\) \\
                 #1 \\
                 \{ \(\rettres\) \} \\
               \end{tabular}
             }
           }

\newcommand{\consttname}{\color{blue} \mathbf{ConstT}}
\newcommand{\consttres}{\vt}
\newcommand{\constt}{\consttname}

\newcommand{\consttruleblock}[1]
           {
             \truleblock{\(\constt\)}{#1}
           }

\newcommand{\globaltname}{\color{blue} \mathbf{GlobalT}}
\newcommand{\globaltargs}{\GN, \TN}
\newcommand{\globaltres}{\pt,\vt,\overline{\lt}}
\newcommand{\globalt}{\globaltname(\globaltargs)}

\newcommand{\globaltruleblock}[1]
           {
             \truleblock{\(\globalt\)}{#1}
           }

\newcommand{\dealloctname}{\color{blue} \mathbf{DeallocT}}
\newcommand{\dealloctargs}{\PCT, \TN}
\newcommand{\dealloctres}{\PCT',\vt,\overline{\lt}}
\newcommand{\dealloct}{\dealloctname(\dealloctargs)}

\newcommand{\freetname}{\color{blue} \mathbf{FreeT}}
\newcommand{\freetargs}{\PCT, \vt}
\newcommand{\freetres}{\PCT',\pt,\vt,\overline{\lt}}
\newcommand{\freet}{\freetname(\freetargs)}

\newcommand{\picasttname}{\color{blue} \mathbf{PICastT}}
\newcommand{\picasttargs}{\PCT, \pt, \vt, \overline{\lt}}
\newcommand{\picasttres}{\vt}
\newcommand{\picastt}{\picasttname(\picasttargs)}

\newcommand{\ipcasttname}{\color{blue} \mathbf{IPCastT}}
\newcommand{\ipcasttargs}{\PCT, \vt_1, \vt_2, \overline{\lt}}
\newcommand{\ipcasttres}{\pt}
\newcommand{\ipcastt}{\ipcasttname(\ipcasttargs)}

\newcommand{\ppcasttname}{\color{blue} \mathbf{PPCastT}}
\newcommand{\ppcasttargs}{\PCT, \pt, \vt, \overline{\lt}}
\newcommand{\ppcasttres}{\pt'}
\newcommand{\ppcastt}{\ppcasttname(\ppcasttargs)}

\newcommand{\iicasttname}{\color{blue} \mathbf{IICastT}}
\newcommand{\iicasttargs}{\PCT, \vt_1}
\newcommand{\iicasttres}{\pt}
\newcommand{\iicastt}{\iicasttname(\iicasttargs)}

\newcommand{\splittname}{\color{blue} \mathbf{SplitT}}
\newcommand{\splittargs}{\PCT, \vt, L}
\newcommand{\splittres}{\PCT'}
\newcommand{\splitt}{\splittname(\splittargs)}

\newcommand{\splittruleblock}[1]
           {
             \truleblock{\(\splitt\)}{#1}
           }

\newcommand{\exprsplittname}{\color{blue} \mathbf{ExprSplitT}}
\newcommand{\exprsplittargs}{\PCT, \vt}
\newcommand{\exprsplittres}{\PCT'}
\newcommand{\exprsplitt}{\exprsplittname(\exprsplittargs)}

\newcommand{\exprsplittruleblock}[1]
           {
             \colorbox{blue!10}{
               \begin{tabular}{l}
                 \(\exprsplitt\) \\
                 #1 \\
                 \{ \(\exprsplittres\) \} \\
               \end{tabular}
             }
           }


\newcommand{\exprjointname}{\color{blue} \mathbf{ExprJoinT}}
\newcommand{\exprjointargs}{\PCT, \PCT', \vt}
\newcommand{\exprjointres}{\PCT'', \vt'}
\newcommand{\exprjoint}{\exprjointname(\exprjointargs)}

\newcommand{\exprjointruleblock}[1]
           {
             \colorbox{blue!10}{
               \begin{tabular}{l}
                 \(\exprjoint\) \\
                 #1 \\
                 \{ \(\exprjointres\) \} \\
               \end{tabular}
             }
           }

\newcommand{\labeltname}{\color{blue} \mathbf{LabelT}}
\newcommand{\labeltargs}{\PCT, \LN}
\newcommand{\labeltres}{\PCT'}
\newcommand{\labelt}{\labeltname(\labeltargs)}

\newcommand{\labeltruleblock}[1]
           {
             \colorbox{blue!10}{
               \begin{tabular}{l}
                 \(\labelt\) \\
                 #1 \\
                 \{ \(\labeltres\) \} \\
               \end{tabular}
             }
           }

\newcommand{\extcalltname}{\color{blue} \mathbf{ExtCallT}}
\newcommand{\extcalltargs}{\PCT, \FN, \FN[f'], \overline{\vt}}
\newcommand{\extcalltres}{\PCT'}
\newcommand{\extcallt}{\extcalltname(\extcalltargs)}

\newcommand{\fieldtname}{\color{blue} \mathbf{FieldT}}
\newcommand{\fieldtargs}{\pt, \TN, \GN}
\newcommand{\fieldtres}{\pt'}
\newcommand{\fieldt}{\fieldtname(\fieldtargs)}

%%%%%%%%%%%%%%%%%

\newcommand{\caseofthree}[7]
           {
             \begin{tabular}{l c l}
               \multicolumn{3}{l}{\textnormal{case } #1 \textnormal{ of}} \\
               #2 & \(\Rightarrow\) & #3 \\
               #4 & \(\Rightarrow\) & #5 \\
               #6 & \(\Rightarrow\) & #7 \\
             \end{tabular}
           }

\newcommand{\caseoftwo}[5]
           {
             \begin{tabular}{l c l}
               \multicolumn{3}{l}{\textnormal{case } #1 \textnormal{ of}} \\
               #2 & \(\Rightarrow\) & #3 \\
               #4 & \(\Rightarrow\) & #5 \\
             \end{tabular}
           }
